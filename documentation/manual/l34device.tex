\chapter{Using cochlear implants from Cochlear}
\label{sec:L34}

To directly stimulate subjects using cochlear implants from Cochlear, \apex uses the NICv2 interface, provided by Cochlear.
This interface is abstracted as the L34Device (to be used under \element{devices}).
It is always called L34, irrespective of the device that is used
with NICv2 (which can be an L34 but also an SP12 or other devices).

\section{L34 setup}

To use \apex with the L34 device, you need the L34 plugin, which is provided on request, provided you sign the Cochlear NIC agreement.

After obtaining the L34 plugin (in the form of the file \filename{l34plugin.dll}), copy it to the \filename{plugins} directory under the main APEX directory.
You also need a properly setup Cochlear NICv2 environment. We refer to the NICv2
documentation on how to do this. For \apex it boils down to having
the right version of NICv2 (the same \apex was compiled against)
and having the Nucleus NIC binaries directory in your path. If
this part is setup correctly, you should be able to use the L34
device with device number 0 (the simulated device).

Before starting an \apex experiment that uses the L34device, make sure the appropriate devices are connected to the computer and the NIC environment is setup correctly.


\section{Electrical stimulation files}

When using the L34 device, you need prefined stimulation patterns.
Electrical stimulation patterns are not stored in the wave format, such as for the wavdevice, but as one of the following file types:

\todo{replace qic by aseq}

\begin{description}
\item[qic] In a qic file, 2 parameters are specified per stimulus:
the channel to be used and the magnitude (in \%). Before a qic
file can be sent to the NICv2 interface, it thus first has to be
mapped by apex.
\item[qicext] electrode, magnitude (CU), period, phasewidth, phasegap
\item[xml] defined by cochlear (ref?)
\end{description}

Either of these files can be generated using the Nucleus Matlab Toolbox, provided by cochlear. The most important difference between xml stimulation files and other simulation files, is that for the latter \apex will do channel mapping using a user map defined in the experiment file.

\subsection{Generating .qic files}
In \filename{.qic} files, pulses are defined by a certain magnitude (between 0
and 1) and a channel. They can be generated by the Matlab function
\matlabcmd{genstimulus\_elec.m} and written to disk using the \matlabcmd{save\_sequence.m} function. A .qic file can be read back into matlab using the \matlabcmd{read\_qic.m} function.

As the period is constant for an entire qic file, powerup frames have to be inserted in the file if no stimulation is desired at a certain instant.

\index{Qic files}


\section{Parameters of the L34 device}

To use the L34 device, define a \element{device} under \element{devices}, use the \attribute{xsi:type="apex:L34DeviceType"} attribute.

The L34 device has some main parameters:
\begin{description}
\item[device\_id] The device number to be used, as defined by Cochlear. E.g., the first L34 connected to the computer has id 1. There is a simulated device with id 0, which allows to debug experiments without having an L34 connected. Note that triggering does not work properly with the simulated device.
\item[implant] The implant can be either \xml{cic3} (Esprit3G or Sprint or
Freedom with old internal part) or \xml{cic4} (Freedom with new internal
part).
\end{description}

When using \filename{.qic} or \filename{.qicext} files, \apex calculates the number of current units on the basis of the
defined map. The map can be either entered directly in the experiment file (inline) or
derived from the Nucleus fitting software R126 (fromR126).
The following map parameters should be defined: number of electrodes,
mode, pulsewidth, pulsegap, period (1/total rate) and the C's and
T's for the different channels.

Example:
\begin{lstlisting}
<devices>
 <device id="l34"  xsi:type="apex:L34DeviceType">
    <device_id>1</device_id>
    <implant>cic3</implant>

    <defaultmap>
    <inline>
    <number_electrodes>3</number_electrodes>
    <mode>MP1+2</mode>
    <pulsewidth>25</pulsewidth>
    <pulsegap>8</pulsegap>
    <period>138.9</period>
    <channel number="1" electrode="22" threshold="120" comfort="150"/>
    ...
    <channel number="22" electrode="1" threshold="130" comfort="152"/>
    </inline>
 </defaultmap>
</device>
<devices>
\end{lstlisting}


Note that Cochlear numbers the electrodes from 22 (apex) to 1 (base). You can use either convention in your \filename{.qic} file, as long as the above map is specified correctly.

Note also that when using stimulation files in the XML format, the map is not used.

\index{L34-device}
\index{Implant}
\index{Mode}
\index{Defaultmap}
\index{Inline}
\index{pulsewidth}
\index{Pulsegap}
\index{Period}





\section{Bilateral stimulation}
For bilateral stimulation, two devices with each a map need to be defined.
To control the order of starting both devices, you can specify a master device. If a master device is specified, this will be the last device started. If not specified, the devices are started in order of appearance in the experiment file. The device that sends a trigger signal will typically be the master device.

The \element{trigger} element can be used to specify trigger signals to be sent or received. The \xml{<trigger>out</trigger>} specifies a trigger out signal to be sent and the \xml{<trigger>in</trigger>} tells the device to wait for a trigger signal before starting stimulation.

Example:
\begin{lstlisting}
<master>l34-1</master>
<device id="l34-1"  xsi:type="apex:L34DeviceType">
	<device_id>1</device_id>
	<trigger>out</trigger>
	...
</device>
<device id="l34-2"  xsi:type="apex:L34DeviceType">
	<device_id>2</device_id>
	<trigger>in</trigger>
	...
</device>
\end{lstlisting}


\index{Bilateral stimulation}
\index{L34}

Note that datablocks are attached to a device, and need to be defined twice if they are to be used with both devices.


\index{Master}
\index{Device}
\index{Trigger}
