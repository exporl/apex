\chapter{Customizing appearance}
\label{sec:Customizing appearance}

The appearance of virtually every element on the screen can be
customized. On a small scale, every element on screen can be
customized by using tags such as bgcolor or fontsize per element,
but on a broader scale every element can be customized by adding a
style element and specifying a style sheet to be used.

Internally the Qt Stylesheet mechanism is used. For a complete overview on how this works, we refer to the Qt manual, which can also be found on the internet at \url{http://doc.trolltech.com/4.3/stylesheet.html}.

Basically, a style sheet consists of a \emph{selector} and a series of properties that can be set.

\begin{lstlisting}
QPushButton#button1 {
    background-color: red;
}
\end{lstlisting}


Here the selector is on line  1 and the property background-color is set to red.

A selector again consists of two parts: the type of elements that will be affected and the name of the concrete element to select. Here the type is QPushButton and the name is button1. The latter name would be what you specify as an id in the experiment file.

All elements used in Apex are listed in table~\ref{tab:elements}. Several generic names of element are given in table~\ref{tab:names}.

In Apex, stylesheets can be specified at 3 levels: inside the \lstinline!<screens>! element in the \lstinline!<apex_style>! element: if placed here, every element on screen will be affected, including the panel buttons and the main menu. If placed in the \lstinline!<style>! element, the stylesheet will affect all screens, but not the panel or menus. If placed in a specific screen element, only that element will be affected.

\begin{table}
\begin{center}
% use packages: array,booktabs
\begin{tabular}{lll}        \toprule
Element name & Screen element & Description \\      \midrule
QPushButton & button & clickable button \\
QLabel & label & text label \\
& picture & picture  \\
\bottomrule
\end{tabular}
\end{center}
\caption{Screen elements for use in style sheets.}
\label{tab:elements}
\end{table}


\begin{table}
\begin{center}
% use packages: array,booktabs
\begin{tabular}{lll}    \toprule
Name & Type & Description \\ \midrule
background & QWidget & The background of the screen \\
startbutton & QPushButton & The start button \\
\bottomrule
\end{tabular}
\end{center}

\caption{Screen element names for use in style sheets.}
\label{tab:names}
\end{table}


The color names that can be used, are those specified in the css standard. A list can be found at \url{http://www.w3schools.com/css/css_colornames.asp}.
