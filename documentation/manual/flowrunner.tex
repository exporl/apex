\chapter{Flowrunner}
\label{sec:flowrunner}

A flowrunner file (extension: \filename{.apf}) is a HTML-file that enables to perform different \apex experiments from one file. The file consists of a JavaScript-script containing separate functions for each \apex experiment you want to run from the flowrunner; these can be called with e.g. buttons.
% experiment function syntax

Example:
\begin{lstlisting}
<html>
        <head>
                <title>Flow runner</title>

                <script src="apex:rtresults.js"></script>

                <script>
                function callback(filename) {
                        var xml = api.getFile(filename);
                }
                api.savedFile.connect(callback);

                function exp1() {
                        api.runExperiment("apex:examples/procedure/extrasimple.apx");
                }
                function exp2() {
                        api.runExperiment("../../examples/procedure/adaptive-1up-1down.apx");
                }
                function exp3() {
                        expressions = {
                        "apex:apex/results[1]/subject[1]": "Piet Snot"
                };
                        api.runExperiment("apex:examples/interactive/extrasimple.apx", expressions);
                }
                function exp4() {
                        api.runExperiment("apex:examples/procedure/extrasimple.apx", {}, "results.apr");
                }
                </script>
        </head>
        <body>
                <button onclick="exp1()">Extra simple</button>
                <button onclick="exp2()">Adaptive</button>
                <button onclick="exp3()">Interactive</button>
                <button onclick="exp4()">Extra simple with results</button>
        <body>
</html>
\end{lstlisting}

It is important that in each experiment file \xml{exitafter} is set to \xml{false} (\xml{<exitafter>false</exitafter>}).